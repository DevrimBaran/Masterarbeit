\chapter{Implementation}\label{ch:implementation}

After the wait-free algorithms got identified and analyzed the next sub-goal to find the best wait-free algorithms is to implement them like already said in \cref{sec:objective}. The algorithms got implemented and published in the accompanying GitHub repository in \cite{githubMA}, where complete implementation of the algorithms can be found. This chapter will present the implementation details of the concurrent queue algorithms seen in \cref{ch:choosing-the-optimal-wait-free-data-structure} in Rust, focusing on the adaptations necessary for \ac{IPC} over shared memory. While the algorithmic logic of each queue has been discussed already in \cref{ch:choosing-the-optimal-wait-free-data-structure}, the implementation required slight deviations to support \ac{IPC} over shared memory. This includes ensuring correct cache allignment, correctly implementing various atomic primitives, and correctly implementing the logic for shared memory support based on short example snippets of the actual rust implementation from the GitHub repository or some general examples. This does not include showing again the same algorithmic logic just in a actual programming language instead of pseudocode, since that would be redundant. This chapter will give the necessary details to understand how to implement the algorithms in Rust and how to adapt them for \ac{IPC} over shared memory.

\section{Shared Memory Management for Inter-Process Communication}

\ac{IPC} through shared memory requires slightly different approaches compared to multi-threaded heap-based communication. The primary constraint is that shared memory regions may be mapped at different virtual addresses in different processes, requiring all data structures to be completely position-independent. Additionally, dynamic memory allocation is not possible within shared memory regions, necessitating buffer allocation from a pre-allocated memory pool for all dynamic data structures. Threads on the other hand can use the process-private heap for dynamic memory allocation, which' memory layout is shared between threads on the same process, but not processes.  

\subsection{Shared Memory Size Calculation}

Each queue provides a method to calculate the exact shared memory size required. This calculation determines how much memory to allocate from the operating system. The following examples from BQueue and \ac{wCQ} demonstrate both simple and complex size calculations as needed:

\begin{lstlisting}[language=Rust, style=boxed, caption={Shared memory size calculation methods}, label={lst:size-calculation}]
// From BQueue - simple calculation
pub const fn shared_size(capacity: usize) -> usize {
    mem::size_of::<Self>()                              // Queue structure
        + capacity * mem::size_of::<MaybeUninit<T>>()   // Data storage
        + capacity * mem::size_of::<AtomicBool>()       // Validity flags
}

// From WCQueue - complex layout calculation
fn layout(num_threads: usize, num_indices: usize) -> (Layout, [usize; 4]) {
    let ring_size = num_indices.next_power_of_two();
    let capacity = ring_size * 2;
    
    let root = Layout::new::<Self>();
    let (l_aq_entries, o_aq_entries) = root
        .extend(Layout::array::<EntryPair>(capacity).unwrap())
        .unwrap();
    let (l_fq_entries, o_fq_entries) = l_aq_entries
        .extend(Layout::array::<EntryPair>(capacity).unwrap())
        .unwrap();
    let (l_records, o_records) = l_fq_entries
        .extend(Layout::array::<ThreadRecord>(num_threads).unwrap())
        .unwrap();
    let (l_final, o_data) = l_records
        .extend(Layout::array::<DataEntry<T>>(num_indices).unwrap())
        .unwrap();
    
    (l_final.pad_to_align(), [o_aq_entries, o_fq_entries, o_records, o_data])
}

pub fn shared_size(num_threads: usize) -> usize {
    let (_layout, _offsets) = Self::layout(num_threads, num_indices);
    _layout.size()  // Total bytes needed for mmap
}
\end{lstlisting}

The size calculation accounts for all components including alignment padding. Some queues like \ac{wCQ} use Rust's \texttt{Layout} API shown in lines 9 to 26 to ensure proper alignment and correctly calculated offsets. For simple queues like BQueue, manual calculation like in lines 2 to 6 suffices.

\subsection{Shared Memory Allocation}

Once the required size is calculated, the following functions in \cref{lst:map-shared} used in all benchmarks demonstrates how to allocate and deallocate the needed shared memory regions using \texttt{mmap}:

\begin{lstlisting}[language=Rust, style=boxed, caption={Shared memory allocation using mmap}, label={lst:map-shared}]
unsafe fn map_shared(bytes: usize) -> *mut u8 {
    let ptr = libc::mmap(
        ptr::null_mut(),
        bytes,                                    // Size from shared_size()
        libc::PROT_READ | libc::PROT_WRITE,
        libc::MAP_SHARED | libc::MAP_ANONYMOUS,   // Shared between processes
        -1,
        0,
    );
    if ptr == libc::MAP_FAILED {
        panic!("mmap failed: {}", std::io::Error::last_os_error());
    }
    ptr.cast()
}

// Cleanup function
unsafe fn unmap_shared(ptr: *mut u8, len: usize) {
    if libc::munmap(ptr.cast(), len) == -1 {
        panic!("munmap failed: {}", std::io::Error::last_os_error());
    }
}
\end{lstlisting}

The \texttt{MAP\_SHARED} flag ensures that modifications to the memory region are visible to all processes that map it. The \texttt{MAP\_ANONYMOUS} flag creates memory not backed by a file. The \texttt{bytes} parameter on line 4 comes directly from the \texttt{shared\_size()} calculation earlier.

\subsection{Memory Layout and Initialization}

After allocating the shared memory region the components needed by the queue's implementations need to be initialized. All queue implementations follow a consistent pattern for shared memory initialization. The initialization function receives the pre-allocated memory pointer and organizes components within that memory region. The most complex example, the WCQueue from \cref{subsubsec:wcq}, demonstrates how multiple components are laid out in shared memory with proper alignment:

\begin{lstlisting}[language=Rust, style=boxed, caption={Memory layout initialization in WCQueue}, label={lst:wcqueue-init}]
pub unsafe fn init_in_shared(mem: *mut u8, num_threads: usize) -> &'static mut Self {
    let mut current_offset = 0usize;
    
    // Align and place queue structure
    current_offset = (current_offset + self_align - 1) & !(self_align - 1);
    let q_ptr = mem.add(current_offset) as *mut Self;
    current_offset += mem::size_of::<Self>();
    
    // Align and place entry arrays
    current_offset = (current_offset + entry_align - 1) & !(entry_align - 1);
    let aq_entries = mem.add(current_offset) as *mut EntryPair;
    current_offset += capacity * mem::size_of::<EntryPair>();
    
    // Initialize structures in-place
    ptr::write(q_ptr, Self {
        aq_entries_offset: current_offset,
        base_ptr: mem,  // Store base pointer for offset calculations
        // Store offsets instead of pointers
    });
    
    &mut *q_ptr
}
\end{lstlisting}

The alignment calculation in line 5 ensures that each component starts at a properly aligned address. This is crucial for atomic operations, which often require natural alignment. The queue structure stores offsets relative to the base pointer rather than absolute pointers in line 16, ensuring position independence. When accessing these components later, the offset is added to the base pointer, as demonstrated in \cref{lst:position-independent}.

\begin{lstlisting}[language=Rust, style=boxed, caption={Position-independent component access}, label={lst:position-independent}]
unsafe fn get_entry(&self, entries_offset: usize, idx: usize) -> &EntryPair {
    let entries = self.base_ptr.add(entries_offset) as *const EntryPair;
    &*entries.add(idx)
}
\end{lstlisting}

\subsection{Node Allocation from Pre-allocated Memory Pools}

Some queues as seen in \cref{ch:choosing-the-optimal-wait-free-data-structure} use dynamic memory allocation within the process-private heap. For the \ac{IPC} over shared memory use-case of this thesis that is not possible, because in shared memory dynamic memory allocations or deallocations with \texttt{malloc} or \texttt{free} are not possible, since that allocates from process-private heaps that are not available for other processes. Therefore, all queues that would normally allocate memory dynamically have been adapted to allocate from the initialized pre-allocated memory pool. As an example the Jiffy Queue algorithm from \cref{subsubsec:jiffy-mpsc-queue} shows a dynamic heap allocation method to allocate new buffers to insert them into the linked-list. This was adapted to \ac{IPC} over shared memory in rust by a pool allocation system with free lists that is similarly impülemented for all other queues needing dynamic memory allocation, as shown in \cref{lst:pool-allocation}.

\begin{lstlisting}[language=Rust, style=boxed, caption={Lock-free memory pool allocation}, label={lst:pool-allocation}]
unsafe fn alloc_node_array_slice(&self) -> *mut Node<T> {
    // Try reuse from free list first
    loop {
        let free_head = self.node_array_slice_free_list_head.load(Ordering::Acquire);
        if free_head.is_null() {
            break;
        }
        
        let next_free = (*(free_head as *mut AtomicPtr<Node<T>>))
            .load(Ordering::Acquire);
            
        if self.node_array_slice_free_list_head.compare_exchange(
            free_head, 
            next_free, 
            Ordering::AcqRel, 
            Ordering::Acquire
        ).is_ok() {
            return free_head;
        }
    }
    
    // Allocate from pre-allocated pool
    let nodes_needed = self.buffer_capacity_per_array;
    let start_idx = self.node_arrays_next_free_node_idx
        .fetch_add(nodes_needed, Ordering::AcqRel);
        
    if start_idx + nodes_needed > self.node_arrays_pool_total_nodes {
        self.node_arrays_next_free_node_idx
            .fetch_sub(nodes_needed, Ordering::Relaxed);
        return ptr::null_mut(); // Pool exhausted
    }
    
    self.node_arrays_pool_start.add(start_idx)
}
\end{lstlisting}

This implementation maintains a free list using \ac{CAS} operations in lines 12 to 16. When the free list is empty, it falls back to allocate from the pre-allocated pool seen in lines 23 and 24. The \texttt{fetch\_add} operation atomically increments the allocation index, ensuring process-safe allocation. If the pool is exhausted, the operation fails by returning a null pointer in line 30.

\section{Cache Line Optimization}
Processors transfer data between cores in cache line units explained in \cref{subsubsec:lamport-circular-buffer-queue}. When multiple processes access data on the same cache line, even if different variables, the cache coherence protocol causes the cache line to bounce between cores, degrading execution times.

\subsection{Explicit Cache Line Padding}

In \cref{ch:choosing-the-optimal-wait-free-data-structure} multiple queues were shown that describe seperating the cache line. To show how this is done in rust the \ac{BLQ} explained in \cref{subsec:single-producer-and-single-consumer} will be taken to show how to explicitly seperate the cache lines, as shown in \cref{lst:cache-separation}.

\begin{lstlisting}[language=Rust, style=boxed, caption={Cache line separation in BlqQueue}, label={lst:cache-separation}]
const CACHE_LINE_SIZE: usize = 64;

#[repr(C)]
#[cfg_attr(any(target_arch = "x86_64", target_arch = "aarch64"), repr(align(64)))]
pub struct SharedIndices {
    pub write: AtomicUsize,  // Producer's cache line
    pub read: AtomicUsize,   // Consumer's cache line
}

#[repr(C, align(64))]
struct ProducerPrivate {
    read_shadow: usize,      // Local copy to avoid false sharing
    write_priv: usize,       // Producer-only write position
}

#[repr(C, align(64))]
struct ConsumerPrivate {
    write_shadow: usize,     // Local copy to avoid false sharing
    read_priv: usize,        // Consumer-only read position
}
\end{lstlisting}

The \texttt{\#[repr(C)]} attribute ensures C-compatible memory layout, while \newline \texttt{\#[repr(align(64))]} forces the structure to start at a cache line boundary shown in lines 4 and 10. Although \texttt{SharedIndices} contains only two \texttt{usize} values with 16 bytes total, the alignment ensures they reside in separate cache lines. The producer updates \texttt{write} while the consumer updates \texttt{read}, preventing false sharing. The shadow copies \texttt{read\_shadow} in line 12 and \texttt{write\_shadow} in line 18 ensures that producer and consumer work on different cache lines, preventing the cache lines to bounce between the producer process and consumer process.

Similarly all queues that need this kind of cache line separation use this pattern to ensure that the producer and consumer do not share cache lines, preventing false sharing leading to cache lines bouncing between cores.

\subsection{Manual Padding Arrays}

For structures where alignment alone is insufficient, manual padding arrays provide a solution, as demonstrated in \cref{lst:manual-padding}. This is used in all queues needing manual padding. As an example the implementation of the David queue explained in \cref{subsubsec:david-queue} uses manual padding.

\begin{lstlisting}[language=Rust, style=boxed, caption={Manual padding for exact cache line control}, label={lst:manual-padding}]
#[repr(C, align(64))]
struct FetchIncrement {
    value: AtomicUsize,
    _padding: [u8; CACHE_LINE_SIZE - std::mem::size_of::<AtomicUsize>()],
}

#[repr(C, align(64))]
struct Node<T> {
    val: Option<T>,
    next: AtomicPtr<Node<T>>,
    _padding: [u8; CACHE_LINE_SIZE - 24], // Fill remaining cache line
}
\end{lstlisting}

The padding array size is calculated to fill the remainder of the cache line seen in lines 4 and 11. For \texttt{FetchIncrement}, the \texttt{AtomicUsize} occupies 8 bytes, so 56 bytes of padding complete the 64-byte cache line. This ensures each \texttt{FetchIncrement} instance occupies exactly one cache line, preventing false sharing in arrays of such structures, as required by the DavidQueue implementation from \cref{subsubsec:david-queue}.

\section{Atomic Primitives Implementation}

The algorithms in \cref{ch:choosing-the-optimal-wait-free-data-structure} all use different kind of atomic primitives. To implement them rust provides a set of atomic operations with explicit memory ordering semantics, allowing control over synchronization order. This section shows how in general the atomic operations from \cref{ch:choosing-the-optimal-wait-free-data-structure} are implemented in Rust across all queues, explained with specific examples. In rust atomic primitives can only be called on atomic types. Hence variables that are used in atomic operations must be defined as atomic types.

\subsection{\acf{FAA}}

The rust implementations of DQueue, BQueue and \ac{wCQ} explained in \cref{ch:choosing-the-optimal-wait-free-data-structure} are a good example, to understand how to implement \ac{FAA} called \texttt{fetch\_add(v, ordering)} in rust.

\begin{lstlisting}[language=Rust, style=boxed, caption={Fetch-and-add with different memory orderings}, label={lst:fetch-and-add}]
// From Jiffy - Allocate multiple nodes at once
let start_node_idx = self
    .node_arrays_next_free_node_idx
    .fetch_add(nodes_needed, Ordering::AcqRel);

// From BQueue - private counter with relaxed ordering
let idx = self.next_item.fetch_add(1, Ordering::Relaxed);

// From WCQueue - with sequential consistency for wait-free algorithm
let seqid = self.tail.fetch_add(1, Ordering::SeqCst);
\end{lstlisting}

In rust \ac{FAA} is a method call on atomic types as seen in line 4. In \cref{lst:fetch-and-add} the memory ordering parameter added as the second argument after the value to add determines the synchronization order of \ac{FAA}. \texttt{Ordering::Relaxed} in line 6 provides no synchronization, suitable for private counters. \texttt{Ordering::AcqRel} in line 3 ensures acquire semantics for the read and release semantics for the write, establishing happens-before relationships as required by the DQueue algorithm. \texttt{Ordering::SeqCst} in line 9 provides the strongest guarantees, ensuring a total order across all sequentially consistent operations, necessary for the complex \ac{wCQ}. One simple solution would be to always use \texttt{Ordering::SeqCst} for all operations, but that would reduce the execution times of the algorithms in a significant way. Consequently it is important to analyze the algorithms from \cref{ch:choosing-the-optimal-wait-free-data-structure} to understand which memory ordering is needed for which operation.

\subsection{\acf{CAS}}

The implementation of \ac{wCQ} shows how to implement \ac{CAS}, called \newline \texttt{compare\_exchange(old\_value, new\_value, ordering\_on\_success, \newline ordering\_on\_failure)} in rust, as shown in \cref{lst:compare-and-swap}.

\begin{lstlisting}[language=Rust, style=boxed, caption={Compare-and-swap variants and usage patterns}, label={lst:compare-and-swap}]
// Strong CAS with sequential consistency (from wcqueue)
match entry.value.compare_exchange(
    packed,
    new_packed,
    Ordering::SeqCst,    // Success ordering
    Ordering::SeqCst,    // Failure ordering
) {
    Ok(_) => {
        fence(Ordering::SeqCst);  // Additional synchronization
        Ok(())
    }
    Err(current) => {
        // Retry with current value
    }
}

// Weak CAS for performance (general example)
match entry.value.compare_exchange_weak(
    old_value,
    new_value,
    Ordering::AcqRel,
    Ordering::Acquire,
) {
    Ok(_) => {
        // do something on success
    }
    Err(current) => {
        // do something on failure
    }
}
\end{lstlisting}

The weak variant \texttt{compare\_exchange\_weak} in beginning at 17 may fail spuriously even when the values match, but can be more efficient on some architectures. The strong variant guarantees success when values match. The two ordering parameters in line 5 and 6 and 21 and 22 specify synchronization order for success and failure cases respectively at lines 8 and 12. This directly implements the CAS operations described in multiple algorithms.

\subsection{Swap Operations}

Swap unconditionally replaces a value and returns the previous value, implemented as \texttt{swap(new\_value, ordering)}, as shown in \cref{lst:swap-operations}. As an example the David Queue and Drescher Queue is used.

\begin{lstlisting}[language=Rust, style=boxed, caption={Unconditional atomic swap operations}, label={lst:swap-operations}]
// From David Queue - unconditional slot update
unsafe fn swap(&self, new_val: usize) -> usize {
    self.value.swap(new_val, Ordering::AcqRel)
}

// From Drescher Queue - atomic pointer swap
let prev_tail = self.tail.swap(new_node_ptr, Ordering::AcqRel);

// From DrescherQueue - simpler FAS primitive
let prev_tail_ptr = self.tail.swap(new_node_ptr, Ordering::AcqRel);
(*prev_tail_ptr).next.store(new_node_ptr, Ordering::Release);
\end{lstlisting}

Swap operations are useful when the previous value is needed but the update is unconditional. The DrescherQueue uses swap to implement its simple enqueue operation (line 10), atomically updating the tail pointer and then linking the previous tail to the new node, exactly as specified in the Drescher algorithm in \cref{subsubsec:drescher-mpsc-queue}.

\subsection{Load and Store with Memory Ordering}

Simple loads and stores also require consideration of memory ordering to ensure correct synchronization according to the algorithms in \cref{ch:choosing-the-optimal-wait-free-data-structure}, as demonstrated in the general example of \cref{lst:load-store}.

\begin{lstlisting}[language=Rust, style=boxed, caption={Memory ordering for loads and stores}, label={lst:load-store}]
// Acquire ordering for reading shared state
let tail = self.tail.load(Ordering::Acquire);
if tail > head {
    // Safe to proceed - acquire ensures we see all writes before tail update
}

// Release ordering for publishing updates
unsafe { (*slot.data.get()).write(item); }  // Write data first
self.head.store(new_head, Ordering::Release);  // Then publish

// Sequential consistency for strong synchronization
let val = entry.value.load(Ordering::SeqCst);
entry.value.store(new_val, Ordering::SeqCst);
\end{lstlisting}

The acquire-release pattern is particularly important. A release store in line 9 synchronizes with an acquire load in line 2 of the same location, ensuring that all writes before the release store are visible to any thread that sees the acquire load. This pattern is used in all queues to ensure correct ordering to the producer-consumer relationship, implementing the memory barriers described in algorithms like Lamport Queue (\cref{subsubsec:lamport-circular-buffer-queue}) and others.

\subsection{Memory Fences}

To still ensure correct data ordering without any memory operation rust provide memeory fencec seen in the \ac{wCQ} rust implementation, as shown in \cref{lst:memory-fences}.

\begin{lstlisting}[language=Rust, style=boxed, caption={Explicit memory fence usage}, label={lst:memory-fences}]
// From WCQueue - ensuring visibility across operations
fence(Ordering::SeqCst);
let packed = entry.value.load(Ordering::SeqCst);
let e = EntryPair::unpack_entry(packed);

if condition {
    fence(Ordering::SeqCst);  // Ensure all prior operations complete
    
    match entry.value.compare_exchange_weak(
        packed,
        new_packed,
        Ordering::SeqCst,
        Ordering::SeqCst,
    ) {
        Ok(_) => {
            fence(Ordering::SeqCst);  // Ensure visibility before proceeding
        }
    }
}
\end{lstlisting}

Fences ensure ordering between operations that might not otherwise synchronize. The \ac{wCQ} implementation uses sequential consistency fences (lines 2, 7, 16) to ensure correctness in its complex algorithm. These kind of fences are used in all queue implementations since for \ac{IPC} this was necessary for correrctness. 

\subsection{Versioned \acf{CAS} (Simulating \acf{LL/SC})}

Some algorithms assume \ac{LL/SC} primitives, which x86-64 does not provide. The Jayanti-Petrovic queue from \cref{subsub:jayanti-mpsc-queue} simulates \ac{LL/SC} using versioned \ac{CAS}, as shown in \cref{lst:versioned-cas}.

\begin{lstlisting}[language=Rust, style=boxed, caption={Versioned CAS for LL/SC simulation}, label={lst:versioned-cas}]
#[repr(C)]
struct CompactMinInfo {
    version: u32,    // Version counter for ABA prevention
    ts_val: u16,     // Timestamp value (compressed)
    ts_pid: u8,      // Process ID (compressed)  
    leaf_idx: u8,    // Leaf index (compressed)
}

impl CompactMinInfo {
    fn to_u64(self) -> u64 {
        ((self.version as u64) << 32)
            | ((self.ts_val as u64) << 16)
            | ((self.ts_pid as u64) << 8)
            | (self.leaf_idx as u64)
    }
}

unsafe fn cas_min_info(&self, old_compact: CompactMinInfo, 
                       new_min_info: MinInfo) -> bool {
    // Increment version on every update
    let new_compact = CompactMinInfo::from_min_info(
        new_min_info, 
        old_compact.version + 1
    );
    
    self.compact_min_info.compare_exchange(
        old_compact.to_u64(),
        new_compact.to_u64(),
        Ordering::AcqRel,
        Ordering::Acquire
    ).is_ok()
}
\end{lstlisting}

The version counter in line 3 prevents the ABA problem where a value changes from A to B and back to A between observations. By incrementing the version on every update seen in line 23, even if the logical value returns to a previous state, the version ensures the \ac{CAS} will fail, simulating \ac{LL/SC} semantics as required by the Jayanti-Petrovic algorithm. This is done in every queue rust implementation that requires \ac{LL/SC} semantics.

\subsection{Unsafe Blocks}

Rust's memory safety guarantees to prevent data races by ensuring that either multiple readers or a single writer can access data at any time. However, the wait-free algorithms in \cref{ch:choosing-the-optimal-wait-free-data-structure} require to bypass these restrictions. With the use of \texttt{unsafe} blocks the the Rust compiler gets indicated that the blocks memory safety is handled by the implementation itself. The \texttt{try\_enq\_inner} function of \acsp{wCQ} Rust implementation in \cref{lst:wcq-unsafe} demonstrates why \texttt{unsafe} blocks are necessary.

\begin{lstlisting}[language=Rust, style=boxed, caption={Wait-free synchronization requiring unsafe}, label={lst:wcq-unsafe}]
    // Multiple threads may execute this concurrently
    unsafe fn try_enq_inner(&self, wq: &InnerWCQ, entries_offset: usize,
                           index: usize) -> Result<(), u64> {
        let tail = wq.tail.cnt.fetch_add(1, Ordering::AcqRel);
        let j = Self::cache_remap(tail as usize, wq.capacity);
        
        let entry = self.get_entry(wq, entries_offset, j);
        loop {
            let packed = entry.value.load(Ordering::Acquire);
            let e = EntryPair::unpack_entry(packed);
            
            // Check if slot is available
            if e.cycle < Self::cycle(tail, wq.ring_size) &&
               (e.is_safe || wq.head.cnt.load(Ordering::Acquire) <= tail) &&
               (e.index == IDX_EMPTY || e.index == IDX_BOTTOM) {
                
                // Attempt to claim slot with CAS
                match entry.value.compare_exchange_weak(
                    packed,
                    new_packed,
                    Ordering::SeqCst,
                    Ordering::SeqCst,
                ) {
                    Ok(_) => return Ok(()),
                    Err(_) => continue,  // Retry
                }
            }
        }
    }
    
    // get_entry uses raw pointer arithmetic
    unsafe fn get_entry(&self, _wq: &InnerWCQ, entries_offset: usize, 
                       idx: usize) -> &EntryPair {
        let entries = self.base_ptr.add(entries_offset) as *const EntryPair;
        &*entries.add(idx)  // Dereference raw pointer
    }
\end{lstlisting}

One reason for an \texttt{unsafe} block is raw pointer dereferencing seen in lines 32 to 34. The \texttt{get\_entry} function performs pointer arithmetic on \texttt{base\_ptr} and dereferences the result which is needed for shared memory. The compiler cannot verify that the calculated address is valid or that no data races occur. The Rust compiler also does not allow concurrent access of multiple writer threads or processes. \texttt{try\_enq\_inner} beginning at line 2 is implementated so that multiple processes can simultaneously call it. Each process, atomically increments the tail to get a unique position in line 4, then accesses potentially the same entry due to cache remapping in lines 5 and 7, and finally attempts to modify the entry in line 18. The Rust compiler cannot verify too that this access is safe, so the developer of this implementation has to indicate to the compiler that this is safe by using an \texttt{unsafe} block.

\section{Validation}
To ensure the correctness of the implemented algorithms, unit and miri tests were performed. The unit tests validate the basic functionality of each queue, ensuring that operations like enqueue and dequeue work as expected. Miri tests were used to check for undefined behavior in concurrent scenarios, ensuring that the algorithms behave correctly under extreme contention. This section will generally describe how the tests were implemented and what they validate, without going into the details of each test case.