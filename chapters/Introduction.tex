\chapter{Introduction}\label{ch:introduction}

\section{Motivation}\label{sec:motivation}

In modern manufacturing and automation, control systems must operate under strict timing constraints to function reliably. If a system fails to meet these constraints, unexpected delays can disrupt processes, leading to instability or even hazardous failures in safety-critical environments. For this reason, \ac{RTS} and low-level programming languages like C or Rust are widely used to ensure predictable execution times.

To achieve these strict timing requirements, many real-time applications involve multiple tasks that must run concurrently and share resources efficiently. Without proper synchronization, problems such as data corruption or race conditions can occur leading to unpredictable behavior. Traditional synchronization methods with locks are commonly used to manage access to shared resources by blocking processes so that only one process at a time accesses the shared resource to exchange data in a proper way. However, these blocking mechanisms introduce significant challenges in real-time settings. Since traditional synchronization methods require processes to wait for resource availability, they can lead to increased response times, potential deadlocks, potential process starvations, and potential priority inversions. These delays are unacceptable in systems that require strict timing guarantees. \cite{herlihy1991wait, brandenburg2019multiprocessorrealtimelockingprotocols, kode2024analysisSynchronization}

To overcome these limitations, there is an increasing interest in synchronization techniques without any blocking mechanisms. A lock-free algorithm for instance functions without any locking mechanism thus no blocking. This guarantees that at least one process completes in a finite number of steps, regardless of contention (multiple processes try to access the same shared resource). This property ensures that at least the system will still work even though one process might be lagging. The only problem is that this will not handle starvation since there is no guarantee that every process will finish its task. \cite{kogan2012methodology}

While lock-free algorithms represent a significant improvement, wait-free algorithms guarantee that every operation completes in a finite number of steps, regardless of contention. This property ensures system responsiveness and predictability, which are essential for real-time applications. By eliminating blocking and contention-based delays, wait-free synchronization prevents priority inversion and ensures that high-priority tasks execute without interference. \cite{kogan2012methodology, herlihy1991wait, brandenburg2019multiprocessorrealtimelockingprotocols}

These synchronization mechanisms are particularly important in the context of \ac{IPC}, which plays a crucial role in \ac{RTS}. \ac{IPC} allows processes to exchange data efficiently, but its performance is heavily influenced by the synchronization techniques used. Traditional \ac{IPC} mechanisms, which often rely on blocking some processes, can introduce significant latency and reduce throughput. Wait-free data structures offer a promising alternative by ensuring that communication operations complete within predictable time bounds. However, selecting appropriate wait-free data structures and evaluating their performance in real-time environments remains a challenge. \cite{timnat2014practical, michael1996simple, huang2002improvingWaitFree, pellegrini2020relevancewaitfreecoordinationalgorithms}

To implement these advanced synchronization techniques safely, the choice of programming language is crucial. The Rust programming language provides useful features for implementing real-time synchronization mechanisms. Its ownership model and strict type system prevent data races and enforce safe concurrency. Additionally, Rust offers fine-grained control over system resources, making it a strong candidate for real-time applications that demand both low latency and high reliability. \cite{xu2023rust, sharma2024rustembeddedsystemscurrent}

The concepts and methods introduced here, including \ac{RTS}, \ac{IPC}, synchronization techniques and problems, wait-free synchronization, and the Rust programming language are explored in greater depth in \cref{ch:background}. 

\section{Objective}\label{sec:objective}

The primary goal of this research is to find the best wait-free data structures that can be used to implement a wait-free synchronization for \ac{IPC} through shared memory in \ac{RTS} using Rust. To do so, this study aims to:

\begin{itemize}
\item Identify and analyze existing wait-free synchronization techniques for \ac{IPC} through shared memory for \ac{RTS}.
\item Implement, validate and compare the performance of existing wait-free synchronization mechanisms for \ac{IPC} through a shared memory for real-time scenarios with each other.
\item Choose and analyze which wait-free data structure for \ac{IPC} through shared memory in a real-time setting using Rust is best suited.
\end{itemize}

By addressing these objectives, this work contributes to the field of wait-free synchronization for \ac{IPC} in \ac{RTS} by providing a practical solution with Rust. The insights gained from this research can help improve the reliability and performance of real-time applications across various domains.

\section{Structure of the Thesis}\label{sec:structure-of-the-thesis}