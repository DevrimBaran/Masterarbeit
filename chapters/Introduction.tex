\chapter{Introduction}\label{ch:introduction}
In modern manufacturing and automation, control systems must operate under strict timing constraints to function reliably. If a system fails to meet these constraints, unexpected delays can disrupt processes, leading to instability or even hazardous failures in safety-critical environments. For this reason, \ac{RTS} and low-level programming languages, such as C or Rust, are widely used to ensure predictable execution times.
\section{Motivation}\label{sec:motivation}
To achieve these strict timing requirements, many real-time applications involve multiple tasks that must run concurrently and share resources efficiently. Without proper synchronisation, problems such as data corruption or race conditions can occur, leading to unpredictable behaviour. Traditional synchronisation methods with locks are commonly used to manage access to shared resources by blocking processes, allowing only one process at a time to access the shared resource and exchange data properly. However, these blocking mechanisms introduce difficulties in real-time settings. Since traditional synchronisation methods require processes to wait for resource availability, they can lead to unpredictable response times through potential deadlocks, process starvation, or priority inversion. Such unpredictability is unacceptable in such systems that require strict timing guarantees. \cite{herlihy1991wait, brandenburg2019multiprocessorrealtimelockingprotocols, kode2024analysisSynchronization}

To overcome these limitations, synchronisation techniques without blocking mechanisms are required. A lock-free algorithm, for instance, functions without any locking mechanism, thus avoiding blocking. This guarantees that at least one process completes in a finite number of steps, regardless of contention (when multiple processes attempt to access the same shared resource). This property ensures that the system will still function even if one process is lagging. The only problem is that this does not prevent starvation, since there is no guarantee that every process will finish its task. \cite{kogan2012methodology}

While lock-free algorithms represent an improvement, wait-free algorithms guarantee that every operation completes in a finite number of steps, regardless of contention. This property ensures system responsiveness and predictability, which are essential for defining timing constraints in real-time applications. \cite{kogan2012methodology, herlihy1991wait, brandenburg2019multiprocessorrealtimelockingprotocols}

These synchronisation mechanisms are particularly important in the context of \ac{IPC}, which is needed in \ac{RTS}. \ac{IPC} allows processes to exchange data efficiently, but its performance is heavily influenced by the synchronisation techniques used. Traditional \ac{IPC} mechanisms, which often rely on blocking some processes. Wait-free data structures offer a promising alternative by ensuring that communication operations complete within predictable time bounds. \cite{timnat2014practical, michael1996simple, huang2002improvingWaitFree, pellegrini2020relevancewaitfreecoordinationalgorithms}

To implement the earlier addressed synchronisation techniques properly, the choice of programming language is essential. The Rust programming language provides helpful features for implementing real-time synchronisation mechanisms. Its ownership model and strict type system prevent data races and enforce safe concurrency. Additionally, Rust offers precise control over system resources, making it a suitable choice for real-time applications that require both low latency and high reliability. \cite{xu2023rust, sharma2024rustembeddedsystemscurrent}

The concepts and methods introduced here, including \ac{RTS}, \ac{IPC}, synchronisation techniques and their difficulties, wait-free synchronisation, and the Rust programming language, are explored in greater depth in \cref{ch:background}. 

\section{Objective}\label{sec:objective}
The primary goal of this research is to identify the most effective wait-free data structures for implementing wait-free synchronisation in \ac{IPC} through shared memory in \ac{RTS} using Rust. To do so, this study aims to:

\begin{itemize}
\item Identify and analyse existing wait-free synchronisation techniques for \ac{IPC} through shared memory for \ac{RTS}.
\item Implement, validate, and compare the performance of existing wait-free synchronisation mechanisms for \ac{IPC} through a shared memory for real-time scenarios with each other.
\item Choose and analyse which wait-free data structure for \ac{IPC} through shared memory in a real-time setting using Rust is best suited.
\end{itemize}

\section{Structure of the Thesis}\label{sec:structure-of-the-thesis}
To describe how to achieve this goal, a deeper knowledge base will be provided in \cref{ch:background} to facilitate understanding of the concepts used in this work. Then, in \cref{ch:related-work}, related work leading to concepts needed for this work will be presented. After that, in \cref{ch:methodology}, the methodology for finding the papers, including the wait-free queues, will be explained. Next in \cref{ch:choosing-the-optimal-wait-free-data-structure}, it will be explained which data structure was chosen, and the wait-free queues found will be explained. Afterwards, in \cref{ch:implementation}, it will be explained how to implement the essential details of the queues without explaining their logic again and following that, in \cref{ch:results} the results of the benchmarks will be presented and analysed, and finally, in \cref{ch:conclusion} a conclusion will be drawn and future work will be discussed.
