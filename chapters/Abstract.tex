\cleardoublepage

% Start with German abstract
\begin{otherlanguage}{ngerman}
\chapter*{Kurzfassung}

\addcontentsline{toc}{chapter}{Kurzfassung}

Vorhersehbare und korrekte Interprozesskommunikation (IPC) ist für Echtzeitsysteme von entscheidender Bedeutung, da Verzögerungen, Unvorhersehbarkeit oder inkonsistente Datenstände zu Instabilität und Ausfällen führen können. Traditionelle Synchronisationsmechanismen verursachen Blockierungen, die zu Deadlocks, aushungernden Prozessen oder Prioritätsinversionen führen können, welche zu unvorhersehbaren Antwortzeiten führen. Um diese Herausforderungen zu bewältigen, bietet die wait-free Synchronisation eine Alternative, die den Abschluss von Operationen, wie dem Austausch von Daten zwischen mehreren Prozessen, in einer begrenzten Anzahl von Schritten garantiert und so die Systemreaktionsfähigkeit und -vorhersehbarkeit sicherstellt.

Diese Arbeit untersucht die Nutzung von wait-free Datenstrukturen für IPC in Echtzeitsystemen mit Fokus auf deren Implementierung in Rust. Das Eigentumsmodell und die strengen Nebenläufigkeitsgarantien von Rust machen es besonders geeignet für die Entwicklung von Synchronisationsmechanismen. Diese Arbeit analysiert, implementiert und validiert bestehende wait-free Verfahren für IPC und benchmarkt ihre Leistung in Shared-Memory-Umgebungen anhand der Ausführungszeiten beim Produzieren und Konsumieren einer festgelegten Anzahl von Datenelementen, um optimale Algorithmen für Echtzeit-Anwendungen zu identifizieren.

\vfill
\noindent\textbf{Stichwörter:} Echtzeitsysteme, wait-free Synchronisation, lock-free Synchronisation, Interprozesskommunikation, Rust
\vfill
\end{otherlanguage}
% Then continue with the english one.
\begin{otherlanguage}{english}
\chapter*{Abstract}

\addcontentsline{toc}{chapter}{Abstract}

Predictable and correct \ac{IPC} is essential for \ac{RTS}, where delays, unpredictability, or inconsistent data can lead to instability and failures. Traditional synchronisation mechanisms introduce blocking, which can result in deadlocks, process starvation, or priority inversion, leading to unpredictable response times. To overcome these challenges, wait-free synchronisation provides an alternative that guarantees operation completion, such as the completion of data exchange between multiple processes, within a bounded number of steps, thereby ensuring system responsiveness and predictability.

This thesis examines the application of wait-free data structures for \ac{IPC} in \ac{RTS}, with a focus on their implementation in the Rust programming language. Rust's ownership model and strict concurrency guarantees make it well-suited for developing synchronisation mechanisms. This work analyses, implements, and validates existing wait-free techniques for \ac{IPC}, benchmarking their performance in shared memory environments based on execution times when producing and consuming a predetermined set of data elements to identify optimal algorithms for real-time applications.

\vfill
\noindent\textbf{Keywords:} real-time systems, wait-free synchronisation, lock-free synchronisation, inter-process communication, Rust
\vfill
\end{otherlanguage}
