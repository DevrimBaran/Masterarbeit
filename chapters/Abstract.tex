\cleardoublepage

% Start with German abstract
\begin{otherlanguage}{ngerman}
\chapter*{Kurzfassung}

\addcontentsline{toc}{chapter}{Kurzfassung}

Vorhersehbare und korrekte Interprozesskommunikation (IPC) ist für Echtzeitsysteme von entscheidender Bedeutung, da Verzögerungen, Unvorhersehbarkeit oder inkonsistente Datenstände zu Instabilität und Ausfällen führen können. Traditionelle Synchronisationsmechanismen verursachen Blockierungen, die zu Prioritätsinversionen und erhöhten Antwortzeiten führen. Um diese Herausforderungen zu bewältigen, bietet die wartefreie Synchronisation eine Alternative, die den Abschluss von Operationen, wie dem Austausch von Daten zwischen mehreren Prozessen, in einer begrenzten Anzahl von Schritten garantiert und so die Systemreaktionsfähigkeit und -zuverlässigkeit sicherstellt.

Diese Arbeit untersucht die Nutzung von wait-free Datenstrukturen für IPC in Echtzeitsystemen mit Fokus auf deren Implementierung in Rust. Das Eigentumsmodell und die strengen Nebenläufigkeitsgarantien von Rust machen es besonders geeignet für die Entwicklung latenzarmer und hochzuverlässiger Synchronisationsmechanismen. Diese Arbeit analysiert bestehende wait-free Methoden für IPC in Echtzeitsystemen und bewertet ihre Leistung im Vergleich zu herkömmlichen Synchronisationsmethoden.

\vfill
\noindent\textbf{Stichwörter:} Echtzeitsysteme, wait-free Synchronisation, lock-free Synchronisation, Interprozesskommunikation, Rust
\vfill
\end{otherlanguage}
% Then continue with the english one.
\begin{otherlanguage}{english}
\chapter*{Abstract}

\addcontentsline{toc}{chapter}{Abstract}

Predictable and correct \ac{IPC} is essential for \ac{RTS}, where delays, unpredictability or inconsistent data can lead to instability and failures. Traditional synchronisation mechanisms introduce blocking, leading to priority inversion and increased response times. To overcome these challenges, wait-free synchronisation provides an alternative that guarantees operation completion, such as the completion of data exchange between multiple processes, within a bounded number of steps, ensuring system responsiveness and reliability.

This thesis explores the use of wait-free data structures for \ac{IPC} in \ac{RTS}, focusing on their implementation in Rust. Rust's ownership model and strict concurrency guarantees make it well-suited for developing low-latency and high-reliability synchronisation mechanisms. This work examines existing wait-free techniques for real-time \ac{IPC}, and evaluates their performance against conventional synchronisation methods.

\vfill
\noindent\textbf{Keywords:} real-time systems, wait-free synchronisation, lock-free synchronisation, inter-process communication, Rust
\vfill
\end{otherlanguage}