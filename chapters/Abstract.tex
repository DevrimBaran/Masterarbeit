\cleardoublepage

% Start with German abstracrt
\begin{otherlanguage}{ngerman}
\chapter*{Kurzfassung}

\addcontentsline{toc}{chapter}{Kurzfassung}

Effiziente und vorhersehbare Interprozesskommunikation (IPC) ist für Echtzeitsysteme von entscheidender Bedeutung, da Verzögerungen und Unvorhersehbarkeit zu Instabilität und Ausfällen führen können. Traditionelle Synchronisationsmechanismen wie Mutexe und Semaphore verursachen Blockierungen, die zu Prioritätsinversionen und erhöhten Antwortzeiten führen. Um diese Herausforderungen zu bewältigen, bietet die wartefreie Synchronisation eine Alternative, die den Abschluss von Operationen in einer begrenzten Anzahl von Schritten garantiert und so die Systemreaktionsfähigkeit und -zuverlässigkeit sicherstellt.

Diese Arbeit untersucht die Nutzung von wartefreien Datenstrukturen für IPC in Echtzeitsystemen mit Fokus auf deren Implementierung und Optimierung in Rust. Das Eigentumsmodell und die strengen Nebenläufigkeitsgarantien von Rust machen es besonders geeignet für die Entwicklung latenzarmer und hochzuverlässiger Synchronisationsmechanismen. Die Studie analysiert bestehende wartefreie Techniken, passt sie für IPC in Echtzeitsystemen an und bewertet ihre Leistung im Vergleich zu herkömmlichen Synchronisationsmethoden. Die Ergebnisse tragen zur Entwicklung effizienter und skalierbarer Synchronisationstechniken bei und verbessern die Vorhersagbarkeit von Echtzeitsystemen.

\vfill
\noindent\textbf{Stichwörter:} Echtzeitsysteme, wait-free Sysnchronisation, lock-free Synchronisation, Interprozesskommunikation, rust
\vfill
\end{otherlanguage}
% Then continue with the english one.
\begin{otherlanguage}{english}
\chapter*{Abstract}

\addcontentsline{toc}{chapter}{Abstract}

Efficient and predictable interprocess communication (IPC) is essential for real-time systems, where delays and unpredictability can lead to instability and failures. Traditional synchronization mechanisms, such as mutexes and semaphores, introduce blocking, leading to priority inversion and increased response times. To overcome these challenges, wait-free synchronization provides an alternative that guarantees operation completion within a bounded number of steps, ensuring system responsiveness and reliability.

This thesis explores the use of wait-free data structures for IPC in real-time systems, focusing on their implementation and optimization in Rust. Rust’s ownership model and strict concurrency guarantees make it well-suited for developing low-latency and high-reliability synchronization mechanisms. The study examines existing wait-free techniques, adapts them for real-time IPC, and evaluates their performance against conventional synchronization methods. The findings contribute to the development of efficient and scalable synchronization techniques, improving the predictability of real-time systems.

\vfill
\noindent\textbf{Keywords:} real-time systems, wait-free synchronization, lock-free synchronization, inter-process communication, rust
\vfill
\end{otherlanguage}
